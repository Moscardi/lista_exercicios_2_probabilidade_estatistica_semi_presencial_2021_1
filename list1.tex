\chapter{Lista de Exercícios 1}


\begin{enumerate}[label=\emph{\arabic*})]

	\item Considere o material denominado "Exemplo de Relatório Estatístico – Estatística Descritiva"

	  	  Para os estudantes da disciplina Probabilidade e Estatística (MA70H), turma S09, o censo apurado para a variável "quantidade de disciplinas matriculadas no semestre letivo 2/2020" foi: 8, 5, 5, 7, 9, 7, 8, 9, 6, 8, 6, 6, 7, 7, 4, 3.

	      A turma S09 tem 20 estudantes matriculados em MA70H, mas o levantamento de dados foi feito em apenas 16 estudantes, pois um estudante chegou atrasado à aula de 24/02, quando foi realizado o levantamento de dados, e outros três estudantes não compareceram à aula de 24/02.

	      \begin{enumerate}[label=\emph{\alph*})]

		      \item Classifique a variável;

		      \item Identifique a unidade de medida da variável;
															      														  \item Identifique a escala de medição da variável;
			  
			  \item \label{item_graficos} Para o censo acima, construa as tabelas 1, 2, 3 e 4 e os gráficos 1, 2, 3, 4 e 5, presentes no material acima referido. Não se esqueça de colocar título completo e fonte em cada tabela e em cada gráfico; caso necessário, colocar notas e chamadas no rodapé das tabelas e gráficos;			  
			  
			  \item Interprete os dados e escreva uma conclusão sobre eles, baseando-se nas tabelas e gráficos construídos no item \ref{item_graficos};
			  
			  \item Fazer capa e folha de rosto nas normas de apresentação de trabalhos acadêmicos. 				  

	      \end{enumerate}	

\end{enumerate}



